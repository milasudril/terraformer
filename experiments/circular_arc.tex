\documentclass[a4paper, 11pt]{scrartcl}
\usepackage[utf8]{inputenc}
\usepackage[T1]{fontenc}
\usepackage{lmodern}
\usepackage[intlimits]{amsmath}
\usepackage{fixme}
\usepackage{amssymb}
\renewcommand{\vec}[1]{\boldsymbol{#1}}
\newcommand{\grad}{\nabla}

\title{Circular arc}
\author{Torbjörn Rathsman}

\begin{document}
\maketitle

A point $P$, at the angle $\theta$, on a circular arc with radius $r$ located at $O$ can be found by the the equation
\[
P(\theta) = O + r\left(\cos(\theta) \vec{\hat{x}} +  \sin(\theta) \vec{\hat{y}}\right)
\]

\section{The angle}
Assume that the tangent vectors $\vec{\hat{t}}_0$ and $\vec{\hat{t}}_0$ at the points $P_0$ and $P_1$ corresponding to some unknown angles $\theta_0$ and $\theta_1$ respectively are known. That is

\[
\vec{\hat{t}}_0 = -\sin(\theta_0) \vec{\hat{x}} + \cos(\theta_0) \vec{\hat{y}}
\]
and
\[
\vec{\hat{t}}_1 = -\sin(\theta_1) \vec{\hat{x}} + \cos(\theta_1) \vec{\hat{y}}
\]

Notice that
\[
\begin{bmatrix}
0 & 1\\
-1 & 0
\end{bmatrix}
\begin{bmatrix}
-\sin(\theta)\\
 \cos(\theta)
\end{bmatrix}
=
\begin{bmatrix}
\cos(\theta)\\
 \sin(\theta)
\end{bmatrix}
\]
Therefore, $\cos(\theta)$ and $\sin(\theta)$ can be found by multiplying a tangent vector by the matrix
\[
R = \begin{bmatrix}
0 & 1\\
-1 & 0
\end{bmatrix}
\]

\section{The origin and radius}
Given $P_0$ and $P_1$, $O$ and $r$ should be determined. The system of equations that needs to be solved is
\[
\left\{
 \begin{aligned}
 P_0 &= O + r\left(\cos(\theta_0) \vec{\hat{x}} +  \sin(\theta_0) \vec{\hat{y}}\right) \\
 P_1 &= O + r\left(\cos(\theta_1) \vec{\hat{x}} +  \sin(\theta_1) \vec{\hat{y}}\right) \\
 \end{aligned}
 \right.
\]

First, solve for $r$. By subtracting the equations
\[
 \overrightarrow{P_0P_1} = P_1 - P_0 = r\left(\left(\cos(\theta_1) - \cos(\theta_0)\right)\vec{\hat{x}}
 + \left(\sin(\theta_1) - \sin(\theta_0)\right)\vec{\hat{y}} \right)
\]
Here, $r$ can be found from either the $x$ equation or $y$ equation. For best precision, the component which gives the largest denominator should be uses.

When $r$ is known, $O$ can be found by substitution. That is
\[
\left\{
\begin{aligned}
 O_x &= x_0 - r\cos(\theta_0) = x_1 - r\cos(\theta_1)\\
 O_y &= y_0 - r\sin(\theta_0) = y_1 - r\sin(\theta_1)
\end{aligned}
\right.
\]
If $x_0 - r\cos(\theta_0) \neq x_1 - r\cos(\theta_1)$ or $y_0 - r\sin(\theta_0) \neq y_1 - r\sin(\theta_1)$, this is not a circle.



\end{document}