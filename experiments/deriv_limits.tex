\documentclass[a4paper, 11pt]{scrartcl}
\usepackage[utf8]{inputenc}
\usepackage[T1]{fontenc}
\usepackage{lmodern}
\usepackage[intlimits]{amsmath}
\usepackage{fixme}


\title{Limits of spline derivatives}
\author{Torbjörn Rathsman}

\begin{document}
\maketitle

With $x\in[0, 1]$

\[
\left\{
\begin{aligned}
 y(x) &= Ax^3 + Bx^2 + Cx + D\\
 y(0) &= 1\\
 y(1) &= 0\\
 y^\prime(0) &= a \\
 y^\prime(1) &= b \\
\end{aligned}
\right.
\iff
\left\{
\begin{aligned}
3Ax^2 + 2Bx + C &= y^\prime(x)\\
D &= 1 \\
A + B + C + D &= 0 \\
C &= a \\
3A + 2B + C &= b
\end{aligned}
\right.
\]
\[
 \left\{
\begin{aligned}
A + B + C + D &= 0 \\
3A + 2B + C &= b \\
C &= a \\
D &= 1
\end{aligned}
\right.
\iff
\left\{
\begin{aligned}
A&=b+a+2\\
B&=-b-2a-3\\
C&=a\\
D&=1
\end{aligned}
\right.
\]

Now solve
\[
 3Ax^2 + 2Bx + C < 0
\]

\paragraph{Case $A=B=0$}
If $A = B = 0$, the curve is a straight line, and thus always monotonic. To satisfy boundary conditions, $a = b = -1$. This is also consistent, since $-2 - 2 + 2 = 0$, and $1 + 3 - 3 = 0$. To conclude, $a = b = -1$ is a valid solution.

\paragraph{Case $A=0\land B\neq 0$}

\paragraph{The general case}


\end{document}$