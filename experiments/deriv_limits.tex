\documentclass[a4paper, 11pt]{scrartcl}
\usepackage[utf8]{inputenc}
\usepackage[T1]{fontenc}
\usepackage{lmodern}
\usepackage[intlimits]{amsmath}
\usepackage{fixme}
\usepackage{amssymb}


\title{Limits of spline derivatives}
\author{Torbjörn Rathsman}

\begin{document}
\maketitle

With $x\in[0, 1]$

\[
\left\{
\begin{aligned}
 y(x) &= Ax^3 + Bx^2 + Cx + D\\
 y(0) &= 1\\
 y(1) &= 0\\
 y^\prime(0) &= a \\
 y^\prime(1) &= b \\
\end{aligned}
\right.
\iff
\left\{
\begin{aligned}
3Ax^2 + 2Bx + C &= y^\prime(x)\\
D &= 1 \\
A + B + C + D &= 0 \\
C &= a \\
3A + 2B + C &= b
\end{aligned}
\right.
\]
\[
 \left\{
\begin{aligned}
A + B + C + D &= 0 \\
3A + 2B + C &= b \\
C &= a \\
D &= 1
\end{aligned}
\right.
\iff
\left\{
\begin{aligned}
A&=b+a+2\\
B&=-b-2a-3\\
C&=a\\
D&=1
\end{aligned}
\right.
\]

Now solve
\[
 3Ax^2 + 2Bx + C < 0
\]

\paragraph{Case $A=B=0$}
If $A = B = 0$, the curve is a straight line, and thus always monotonic. To satisfy boundary conditions, $a = b = -1$. This is also consistent with the equations for $A$ and $B$, since $-2 - 2 + 2 = 0$, and $1 + 3 - 3 = 0$. To conclude, $a = b = -1$ is a valid solution.

\paragraph{Case $A=0\land B\neq 0$}
In this case $2Bx + C = -2\left(b + 2a + 3\right)x + a < 0$, since $A = 0$, $b = -(2 + a)$, which implies that $-2\left(b + 2a + 3\right)x + a = -2\left(a + 1\right)x + a < 0$.

First, assume that $a < -1$. In this case, the range of $x$ requires that $2k + a < 0$, where $k = -(a + 1) > 0$. This is true if and only if $a < -2k = 2(a + 1) \iff -a < 2 \iff a > -2$.

Now, assume that $a > -1$. In this case, the range of $x$ requires that $a < 0$, which is trivial.

To conclude, $a\in[-2, 0] \iff b\in[-2, 0]$.


\paragraph{Case $A>0$}
If $A>0$,
\[
x^2 +\frac{2B}{3A}x + \frac{C}{3A} < 0
\]
By completing the square,
\[
 \left(x - \frac{B}{3A}\right)^2 - \left(\frac{B}{3A}\right)^2 + \frac{C}{3A} < 0
\]
Since LHS should be less than zero, its value should be maximized. This happens at $x = 1$. This directly gives that $b < 0$. The solution is only valid when $A = b+a+2 > 0$, which gives $0 > a > -b - 2$. The upper bound comes from the fact that the derivative must be negative also in $x = 0$. The upper bound implies that $b > -2$.

\paragraph{Case $A < 0$} If $A < 0$,
\[
x^2 +\frac{2B}{3A}x + \frac{C}{3A} > 0
\]
By completing the square,
\[
 \left(x - \frac{B}{3A}\right)^2 - \left(\frac{B}{3A}\right)^2 + \frac{C}{3A} > 0
\]
Now LHS should instead be minimized, which happens for $x = \frac{B}{3A}$. Substituting this back into the derivative, gives
\[
 3A\left(\frac{B}{3A}\right)^2 + 2B\left(\frac{B}{3A}\right) + C < 0 \iff B^2 + 2B^2 + 3AC = 3B^2 + 3AC > 0
\]
Then
\[
(b+2a+3)^2 + (b+a+2)a = {{b}^{2}}+ 5 ab + 6b + 5 {{a}^{2}}+14 a+9 > 0
\]
which means that $b$ must be outside the range, unless $a\in[-\frac{4}{5}, 0]$,
\[
 \left[\frac{-\sqrt{5 {{a}^{2}}+4 a}-(5a+6)}{2}, \frac{\sqrt{5 {{a}^{2}}+4 a}-(5a+6)}{2}\right]
\]
Since $A < 0$, it must also hold that $b+a+2 < 0 \iff b < -(a + 2)$. Moreover, since $x\in[0, 1]$, $B = -b-2a-3$ has the same sign as $A$, and $B=-b-2a-3 < 0 \iff b > -(3 + 2a)$. Then $b$ is within the set
\[
 \complement\left[\frac{-\sqrt{5 {{a}^{2}}+4 a}-(5a+6)}{2}, \frac{\sqrt{5 {{a}^{2}}+4 a}-(5a+6)}{2}\right] \cap \left[-(3 + 2a), -(2 + a)\right]
\], with $a < 0$.



\end{document}